

\documentclass{article}
\usepackage[utf8]{inputenc}
\usepackage{setspace}
\usepackage{ mathrsfs }
\usepackage{graphicx}
\usepackage{amssymb} %maths
\usepackage{amsmath} %maths
\usepackage[margin=0.2in]{geometry}
\usepackage{graphicx}
\usepackage{ulem}
\setlength{\parindent}{0pt}
\setlength{\parskip}{10pt}
\usepackage{hyperref}
\usepackage[autostyle]{csquotes}

\usepackage{cancel}
\renewcommand{\i}{\textit}
\renewcommand{\b}{\textbf}
\newcommand{\q}{\enquote}
%\vskip1.0in





\begin{document}

{\setstretch{0.0}{

\begin{huge}

\b{STARSHIP}

This program presents a quick succession of elementary cellular automata. The base (number of cell states) and neighborhood length can be adjusted, so that the effective rule can be as small as 2 symbols or as great as 1024 (for instance.)

Originally Starship was a console app relying on escape codes and textgraphics, and this original code is still present. But the current version uses Raylib for much more detail and speed.

The automata can be seeded with a random first row, or it can be seeded in \q{mountain} mode, switching only the central bit of the top row on and letting this information travel down in a triangular shape. The video demonstrating this program uses shortwave and/or AM radio noise as a soundtrack.


\end{huge}

}}
\end{document}
